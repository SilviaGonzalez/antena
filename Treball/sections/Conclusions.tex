\chapter{Conclusions}

No hi ha res millor que fer un experiment per interioritzar el millor possible conceptes que \textit{a priori} poden semblar més abstractes. Un cop hem construït l'antena i l'hem mesurat i fins i tot provat per escoltar a la torre de control de Sabadell (LELL), ens sentim molt realitzats doncs hem vist amb els nostres propis ulls que amb pocs recursos econòmics es pot construït i fer servir una antena bàsica.

Aquest treball ens ha ajudat a trencar una barrera que podríem tindre en un primer moment degut a que al no ser especialistes en telecomunicacions, ens podria haver semblat com una tasca força llunyana i complexa.

A part, havíem vist baluns abans, però únicament per fóra, i ara hem construït el nostre propi. Això, un cop més, ha sigut revelador. 

Per acabar, s'ha vist que l'antena ha estat dissenyada i construïda satisfactòriament doncs tots els paràmetres indiquen que aquesta compleix adequadament amb els requisits inicials del projecte.

------------------------------------------------------------------------------------------
Still to do:
1-4nec2 con 75 ohms i comprara a resultados experimentales. EL pone:
WPer defecte, el programa 4NEC2 calcula la relació d'ona estacionària (SWR) respecte a
50  és a dir, considera que la línia de transmissió connectada a l'antena te una
impedància característica de 50 Si feu servir un cable coaxial amb una impedància
característica de 75 el valor calculat no és correcte. L'hauríeu de corregir posant Z0 =
75 a l'apartat de "settings"."
2-Cerca bibliogràfica (normatives de OACI, UIT, "Cuadro Nacional de Atribución de
Frecuencias"...). -: basicamente buscar estos docs i referenciarlos en la biblio a alguna parte suitable.
3-Fotos del dia del tejado en el apartado Mesures/Altres
4-comparar cd numeric amb taula entregada¿
5-subir reultados de cd numerico