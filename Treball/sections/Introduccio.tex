\chapter{Introducció}
El present treball complementa les classes teòriques de l'assignatura de Transport Aeri i Sistemes de Radionavegació impartida al màster d'Enginyeria Aeronàutica de la Universitat Politècnica de Catalunya.

Els requisits estableixen que s'ha de construir una antena per un avió d'aviació general amb les següents prestacions:
\begin{itemize}
\item El marge de freqüències d'utilització ha de ser entre els \textbf{108 i els 137MHz}, és a dir, la banda aeronàutica dins de la banda VHF. La freqüència central de banda serà doncs $f_c = 122$MHz.
\item La \textbf{polarització} ha de ser \textbf{horitzontal}.
\item Es recomana que l'antena sigui \textbf{poc directiva} per tal de poder comunicar-se amb diverses estacions aleatòriament situades des de el fuselatge d'un avió.
\item Ha d'aguantar \textbf{10W} en mode de transmissió.
\item S'ha de tindre en compte el \textbf{drag paràsit} que genera, doncs és important que aquest no sigui significatiu. Per aquest tipus d'avions les velocitats màximes es torben sobre els 200kts.
\end{itemize}
A banda, es requereix que l'antena sigui també \textbf{econòmica}.

Vists els requisits i coneixent l'estat de l'art de les antenes emprades en avions, el projecte es centra les següents opcions:
\begin{itemize}
\item Antena dipol en $\lambda/2$
\item Antena monopol en $\lambda/4$
\end{itemize}

A continuació es faran estudis electromagnètics teòrics, s'escollirà una opció i finalment es mostrarà els resultats experimentals de l'opció escollida i construïda.